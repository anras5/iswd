\documentclass[a4paper,12pt]{article}

\usepackage{graphicx} 
\usepackage[T1]{fontenc}
\usepackage{polski}
\usepackage[utf8]{inputenc}
\usepackage[a4paper]{geometry}
\usepackage{xcolor}
\usepackage{float}

\usepackage{graphicx} 
\usepackage[T1]{fontenc}
\usepackage[polish]{babel}
\usepackage[utf8]{inputenc}
\usepackage{caption}
\usepackage[a4paper]{geometry}
\usepackage{amsmath}
\usepackage[section]{placeins}
\usepackage[tagged, highstructure]{accessibility}
\usepackage{tabularx}
\usepackage{pdfcomment}
\usepackage{xcolor}
\usepackage{float}

\author{{Filip Marciniak 148148, Szymon Pasternak 148146}}
\date{{26.05.2024}}

\title{Graniczna Analiza Danych - raport}

\begin{document}
\maketitle

W sprawozdaniu przedstawione zostały wyniki i wnioski z analizy efektywności lotnisk.

\section{Efektywność}

W poniższej tabeli przedstawiono otrzymane wartości efektywności dla analizowanych lotnisk.
\begin{table}[H]
    \centering
    \begin{tabular}{c|c}
    \hline
         Lotnisko & Efektywność  \\ \hline
         WAW & 1.000 \\
         KRK & 1.000 \\
         KAT & 0.591 \\
         WRO & 1.000 \\
         POZ & 0.800 \\
         LCJ & 0.300 \\
         GDN & 1.000 \\
         SZZ & 0.271 \\
         BZG & 1.000 \\
         RZE & 0.409 \\
         IEG & 0.258 \\
    \end{tabular}
    \caption{Wartości efektywności dla analizowanych lotnisk}
    \label{tab:airports-efficiency}
\end{table}

\noindent Lotniska efektywne oraz nieefektywne: \\
Lotniska efektywne: WAW, KRK, WRO, GDN, BZG \\
Lotniska nieefektywne: KAT, POZ, LCJ, SZZ, RZE, IEG

\section{Hipotetyczna jednostka porównawcza oraz potrzebne poprawki}

Otrzymane wartości wejść hipotetycznej jednostki porównawczej poprawki oraz potrzebne do osiągnięcia efektywności (tylko wejścia) dla wszystkich nieefektywnych lotnisk.

\begin{table}[H]
    \centering
    \begin{tabular}{c|cccc|cccc}
    \hline
        & \multicolumn{4}{c|}{HCU} & \multicolumn{4}{c}{Poprawki}  \\
         Lotnisko & $i_1$ & $i_2$ & $i_3$ & $i_4$ & $i_1$ & $i_2$ & $i_3$ & $i_4$ \\ \hline
         KAT & 2.128 & 18.919 & 33.935 & 6.208 & 1.472 & 13.081 & 23.465 & 4.292 \\
         POZ & 1.200 & 7.998 & 19.195 & 3.199 & 0.300 & 2.002 & 4.805 & 0.801 \\
         LCJ & 0.180 & 3.600 & 7.201 & 1.170 & 0.420 & 8.400 & 16.799 & 2.730 \\
         SZZ & 0.190 & 2.708 & 6.959 & 0.514 & 0.510 & 7.292 & 18.741 & 1.386 \\
         RZE & 0.246 & 2.455 & 4.624 & 1.105 & 0.354 & 3.545 & 6.676 & 1.595 \\
         IEG & 0.026 & 2.585 & 16.387 & 0.775 & 0.074 & 7.415 & 47.013 & 2.225 \\
    \end{tabular}
    \caption{Wartości wejść hipotetycznej jednostki porównawczej oraz poprawki potrzebne do osiągnięcia efektywności dla nieefektywnych lotnisk }
    \label{tab:airports-hcu-and-improvements}
\end{table}

\section{Superefektywność}

Otrzymane wartości superefektywności dla wszystkich lotnisk.

\begin{table}[H]
    \centering
    \begin{tabular}{c|c}
    \hline
         Lotnisko & Superefektywność  \\ \hline
         WAW & 2.278 \\
         KRK & 1.124 \\
         KAT & 0.591 \\
         WRO & 1.040 \\
         POZ & 0.800 \\
         LCJ & 0.300 \\
         GDN & 2.000 \\
         SZZ & 0.271 \\
         BZG & 1.746 \\
         RZE & 0.409 \\
         IEG & 0.258 \\
    \end{tabular}
    \caption{Wartości superefektywności dla analizowanych lotnisk}
    \label{tab:airports-super-efficiency}
\end{table}

\section{Efektywność krzyżowa}

W poniższej tabeli przedstawiono macierz efektywności krzyżowych dla wszystkich lotnisk oraz ich średnie efektywności krzyżowe. 

\begin{table}[H]
\resizebox{\textwidth}{!}{
\begin{tabular}{c|ccccccccccc|c}
\hline
& WAW & KRK & KAT & WRO & POZ & LCJ & GDN &SZZ & BZG & RZE & IEG & $CR_{avg}$ \\ \hline
WAW & 1.0 & 0.214 & 0.108 & 0.338 & 0.219 & 0.058 & 0.509 & 0.119 & 0.279 & 0.07 & 0.011 & 0.794 \\
KRK & 0.8 & 1.0 & 0.456 & 0.908 & 0.59 & 0.136 & 0.508 & 0.127 & 0.961 & 0.289 & 0.001 & 0.718 \\
KAT & 0.913 & 1.0 & 0.591209 & 1.0 & 0.774 & 0.259 & 1.0 & 0.238 & 0.973 & 0.409 & 0.002 & 0.383 \\
WRO & 0.997 & 1.0 & 0.47 & 1.0 & 0.649 & 0.153 & 0.615 & 0.156 & 1.0 & 0.295 & 0.003 & 0.756 \\
POZ & 1.0 & 1.0 & 0.563 & 1.0 & 0.799801 & 0.255 & 1.0 & 0.243 & 0.909 & 0.403 & 0.006 & 0.574 \\
LCJ & 0.595 & 0.491 & 0.278 & 0.605 & 0.512 & 0.300036 & 1.0 & 0.261 & 1.0 & 0.273 & 0.078 & 0.213 \\
GDN & 0.452 & 0.468 & 0.333 & 0.5 & 0.433 & 0.25 & 1.0 & 0.214 & 0.5 & 0.25 & 0.025 & 0.812 \\
SZZ & 1.0 & 0.755 & 0.371 & 0.856 & 0.737 & 0.273 & 1.0 & 0.270787 & 1.0 & 0.346 & 0.036 & 0.201 \\
BZG & 0.55 & 0.543 & 0.202 & 0.573 & 0.37 & 0.096 & 0.302 & 0.09 & 1.0 & 0.17 & 0.005 & 0.863 \\
RZE & 0.903 & 0.996 & 0.591 & 1.0 & 0.77 & 0.261 & 1.0 & 0.238 & 1.0 & 0.409183 & 0.002 & 0.287 \\
IEG & 0.523 & 0.428 & 0.248 & 0.531 & 0.458 & 0.297 & 1.0 & 0.254 & 0.876 & 0.247 & 0.258475 & 0.039 \\
\hline
\end{tabular}}
\label{tab:airports-cross-efficiency}
\end{table}
            
\section{Rozkład efektywności}
\textcolor{blue}{W tej sekcji pokaż wyniki rozkładu efektywności (podział na 5 przedziałów) oraz oszacowaną oczekiwaną wartość efektywności dla wszystkich lotnisk.}
\begin{table}[H]
\begin{tabular}{c|ccccc|c}
\hline
    & $[0-0.2)$ & $[0.2-0.4)$ & $[0.4-0.6)$ & $[0.6-0.8)$ & $[0.8-1.0]$ & $EE$    \\ \hline
WAW & 0.00      & \ldots & \ldots & \ldots & \ldots & \ldots \\
\ldots & \ldots & \ldots & \ldots & \ldots & \ldots & \ldots \\
\hline
\end{tabular}
\label{tab:efficiency-distribution}
\end{table}

\section{Rankingi jednostek}
\textcolor{blue}{Przedstaw i porównaj rankingi lotnisk uzyskane różnymi metodami (superefektywność, średnia efektywność krzyżowa oraz oczekiwana wartość efektywności).}

\noindent Superefektywność: $WAW \succ \ldots \succ \ldots$ \\
Średnia efektywność krzyżowa: \ldots \\
Oczekiwana wartość efektywności: \ldots \\

\textcolor{blue}{Tu przedstaw wnioski z porównania rankingów.}

\end{document}