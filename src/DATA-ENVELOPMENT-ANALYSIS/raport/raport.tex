\documentclass[a4paper,12pt]{article}

\usepackage{graphicx} 
\usepackage[T1]{fontenc}
\usepackage{polski}
\usepackage[utf8]{inputenc}
\usepackage[a4paper]{geometry}
\usepackage{xcolor}
\usepackage{float}

\usepackage{graphicx} 
\usepackage[T1]{fontenc}
\usepackage[polish]{babel}
\usepackage[utf8]{inputenc}
\usepackage{caption}
\usepackage[a4paper]{geometry}
\usepackage{amsmath}
\usepackage[section]{placeins}
\usepackage{tabularx}
\usepackage{pdfcomment}

\hypersetup{
    colorlinks,
    linkcolor={red!50!black},
    citecolor={blue!50!black},
    urlcolor={blue!80!black}
}

\author{{Filip Marciniak 148148, Szymon Pasternak 148146}}
\date{{26.05.2024}}

\title{Graniczna Analiza Danych - raport}

\begin{document}
\maketitle

W sprawozdaniu przedstawione zostały wyniki i wnioski z analizy efektywności lotnisk.
W tabeli \ref{tab:airports} przedstawiono dane wejściowe dla problemu.

\begin{table}[H]
    \centering
    \begin{tabular}{c|cccc|cc}
    \hline
         Lotnisko & i1 & i2 & i3 & i4 & o1 & o2  \\ \hline
         WAW & 10.5 & 36 & 129.4 & 7 & 9.5 & 129.7 \\
         KRK & 3.1 & 19 & 31.6 & 7.9 & 2.9 & 31.3 \\
         KAT & 3.6 & 32 & 57.4 & 10.5 & 2.4 & 21.1 \\
         WRO & 1.5 & 12 & 18 & 3 & 1.5 & 18.8 \\
         POZ & 1.5 & 10 & 24 & 4 & 1.3 & 16.2 \\
         LCJ & 0.6 & 12 & 24 & 3.9 & 0.3 & 4.2 \\
         GDN & 1 & 15 & 42.9 & 2.5 & 2 & 23.6 \\
         SZZ & 0.7 & 10 & 25.7 & 1.9 & 0.3 & 4.2 \\
         BZG & 0.3 & 6 & 3.4 & 1.2 & 0.3 & 6.2 \\
         RZE & 0.6 & 6 & 11.3 & 2.7 & 0.3 & 3.5 \\
         IEG & 0.1 & 10 & 63.4 & 3 & 0.005 & 0.61 \\
    \end{tabular}
    \caption{Wartości wejść oraz wyjść dla lotnisk}
    \label{tab:airports}
\end{table}

\noindent W ramach zadania należy przeprowadzić analizę efektywności dla przedstawionego problemu, w którym jest 11 lotnisk, z których każde ma 4 wejścia oraz 2 wyjścia.


\section{Efektywność}

W tabeli \ref{tab:airports-efficiency} przedstawiono otrzymane wartości efektywności dla analizowanych lotnisk.

\begin{table}[H]
    \centering
    \begin{tabular}{c|c}
    \hline
         Lotnisko & Efektywność  \\ \hline
         WAW & 1.000 \\
         KRK & 1.000 \\
         KAT & 0.591 \\
         WRO & 1.000 \\
         POZ & 0.800 \\
         LCJ & 0.300 \\
         GDN & 1.000 \\
         SZZ & 0.271 \\
         BZG & 1.000 \\
         RZE & 0.409 \\
         IEG & 0.258 \\   
    \hline
    \end{tabular}
    \caption{Wartości efektywności dla analizowanych lotnisk}
    \label{tab:airports-efficiency}
\end{table}

\noindent Lotniska efektywne oraz nieefektywne: \\
Lotniska efektywne: WAW, KRK, WRO, GDN, BZG \\
Lotniska nieefektywne: KAT, POZ, LCJ, SZZ, RZE, IEG \\
Wśród jedenastu rozważanych lotnisk, pięć z nich okazało się efektywnych, a sześć nieefektywnych.
Różnice wśród lotnisk nieefektywnych są spore, przykładowo Poznań uzyskuje efektywność na poziomie 0.8, a Zielona-Góra Babimost na poziomie 0.258.
Co można łatwo zauważyć, to że lotniska które znajdują się w pobliżu większych miast oraz metropolii uzyskują wyższe efektywności lub po prostu są efektywne.


\section{Hipotetyczna jednostka porównawcza oraz potrzebne poprawki}

W tabeli \ref{tab:airports-hcu-and-improvements} przedstawione zostały otrzymane wartości wejść hipotetycznej jednostki porównawczej poprawki oraz potrzebne do osiągnięcia efektywności (tylko wejścia) dla wszystkich nieefektywnych lotnisk.

\begin{table}[H]
    \centering
    \begin{tabular}{c|cccc|cccc}
    \hline
        & \multicolumn{4}{c|}{HCU} & \multicolumn{4}{c}{Poprawki}  \\
         Lotnisko & $i_1$ & $i_2$ & $i_3$ & $i_4$ & $i_1$ & $i_2$ & $i_3$ & $i_4$ \\ \hline
        KAT & 2.128 & 18.919 & 33.935 & 4.396 & 1.472 & 13.081 & 23.465 & 6.104 \\
        POZ & 1.200 & 7.998 & 19.195 & 1.928 & 0.300 & 2.002 & 4.805 & 2.072 \\
        LCJ & 0.180 & 2.783 & 7.201 & 0.475 & 0.420 & 9.217 & 16.799 & 3.425 \\
        SZZ & 0.190 & 2.708 & 6.959 & 0.465 & 0.510 & 7.292 & 18.741 & 1.435 \\
        RZE & 0.246 & 2.455 & 4.624 & 0.537 & 0.354 & 3.545 & 6.676 & 2.163 \\
        IEG & 0.026 & 0.388 & 1.109 & 0.065 & 0.074 & 9.612 & 62.291 & 2.935 \\    
    \hline
    \end{tabular}
    \caption{Wartości wejść hipotetycznej jednostki porównawczej oraz poprawki potrzebne do osiągnięcia efektywności dla nieefektywnych lotnisk }
    \label{tab:airports-hcu-and-improvements}
\end{table}

\noindent Największe poprawki można zaobserwować dla lotniska IEG - w szczególności dla wejścia trzeciego, czyli średniej liczby samolotów, które może obsłużyć lotnisko.
Lotnisko w Katowicach jako jedyne ma poprawkę większą niż 1 dla wejścia pierwszego, czyli dla przepustowości przepływu pasażerów. 
Można również zauważyć, że lotnisko w Poznaniu, które miało nie najgorszą wartość efektywności ma małe poprawki w porównaniu do pozostałych lotnisk (w szczególności wejście 2 oraz 3).

\section{Superefektywność}

W tabeli \ref{tab:airports-super-efficiency} przedstawione zostały otrzymane wartości superefektywności dla wszystkich lotnisk.

\begin{table}[H]
    \centering
    \begin{tabular}{c|c}
    \hline
         Lotnisko & Superefektywność  \\ \hline
         WAW & 2.278 \\
         KRK & 1.124 \\
         KAT & 0.591 \\
         WRO & 1.040 \\
         POZ & 0.800 \\
         LCJ & 0.300 \\
         GDN & 2.000 \\
         SZZ & 0.271 \\
         BZG & 1.746 \\
         RZE & 0.409 \\
         IEG & 0.258 \\
    \hline
    \end{tabular}
    \caption{Wartości superefektywności dla analizowanych lotnisk}
    \label{tab:airports-super-efficiency}
\end{table}

\noindent Największą superefektywność posiada Lotnisko Chopina w Warszawie (wartość większa od 2). Następne jest (co może być nieco zaskakujące) lotnisko w Gdańsku. Niewiele, ponieważ jedynie o cztery setne ponad 1.0 podniosła się efektywność dla lotniska we Wrocławiu.

\section{Efektywność krzyżowa}

W tabeli \ref{tab:airports-cross-efficiency} przedstawiono macierz efektywności krzyżowych dla wszystkich lotnisk oraz ich średnie efektywności krzyżowe. 

\begin{table}[H]
    \resizebox{\textwidth}{!}{
    \begin{tabular}{c|ccccccccccc|c}
    \hline
        & WAW & KRK & KAT & WRO & POZ & LCJ & GDN &SZZ & BZG & RZE & IEG & $CR_{avg}$ \\ \hline
        WAW & 1.000 & 0.214 & 0.108 & 0.338 & 0.219 & 0.058 & 0.509 & 0.119 & 0.279 & 0.070 & 0.011 & 0.794 \\
        KRK & 0.800 & 1.000 & 0.456 & 0.908 & 0.590 & 0.136 & 0.508 & 0.127 & 0.961 & 0.289 & 0.001 & 0.718 \\
        KAT & 0.913 & 1.000 & 0.591 & 1.000 & 0.774 & 0.259 & 1.000 & 0.238 & 0.973 & 0.409 & 0.002 & 0.383 \\
        WRO & 0.997 & 1.000 & 0.470 & 1.000 & 0.649 & 0.153 & 0.615 & 0.156 & 1.000 & 0.295 & 0.003 & 0.756 \\
        POZ & 1.000 & 1.000 & 0.563 & 1.000 & 0.800 & 0.255 & 1.000 & 0.243 & 0.909 & 0.403 & 0.006 & 0.574 \\
        LCJ & 0.595 & 0.491 & 0.278 & 0.605 & 0.512 & 0.300 & 1.000 & 0.261 & 1.000 & 0.273 & 0.078 & 0.213 \\
        GDN & 0.452 & 0.468 & 0.333 & 0.500 & 0.433 & 0.250 & 1.000 & 0.214 & 0.500 & 0.250 & 0.025 & 0.812 \\
        SZZ & 1.000 & 0.755 & 0.371 & 0.856 & 0.737 & 0.273 & 1.000 & 0.271 & 1.000 & 0.346 & 0.036 & 0.201 \\
        BZG & 0.550 & 0.543 & 0.202 & 0.573 & 0.370 & 0.096 & 0.302 & 0.090 & 1.000 & 0.170 & 0.005 & 0.863 \\
        RZE & 0.903 & 0.996 & 0.591 & 1.000 & 0.770 & 0.261 & 1.000 & 0.238 & 1.000 & 0.409 & 0.002 & 0.287 \\
        IEG & 0.523 & 0.428 & 0.248 & 0.531 & 0.458 & 0.297 & 1.000 & 0.254 & 0.876 & 0.247 & 0.258 & 0.039 \\
    \hline
    \end{tabular}}
    \caption{Wartości efektywności krzyżowych dla analizowanych lotnisk}
    \label{tab:airports-cross-efficiency}
\end{table}

\noindent Efektywność krzyżowa umożliwia analizę efektywności dla wektorów wag, które są optymalne dla innych lotnisk.
Uśrednione wyniki dla każdego lotniska wskazują, że największe efektywności mają lotniska BZG (Bydgoszcz), GDN (Gdańsk) oraz WAW (Warszawa Okęcie). Najgorzej wypada IEG (Babimost Zielona Góra).
W tabeli można zaobserwować ciekawe wartości dla wag optymalnych dla GDN oraz BZG, dla których sporo lotnisk osiąga efektywność równą 1.
            
\section{Rozkład efektywności}
W tabeli \ref{tab:efficiency-distribution} zaprezentowano wyniki rozkładu efektywności (podział na 5 przedziałów) oraz oszacowana oczekiwana wartość efektywności dla wszystkich lotnisk.

\begin{table}[H]
\begin{tabular}{c|ccccc|c}
\hline
    & $[0-0.2)$ & $[0.2-0.4)$ & $[0.4-0.6)$ & $[0.6-0.8)$ & $[0.8-1.0]$ & $EE$    \\ \hline
    WAW & 0.000 & 0.000 & 0.000 & 0.120 & 0.880 & 0.948 \\
    KRK & 0.000 & 0.010 & 0.240 & 0.660 & 0.090 & 0.669 \\
    KAT & 0.040 & 0.940 & 0.020 & 0.000 & 0.000 & 0.284 \\
    WRO & 0.000 & 0.000 & 0.140 & 0.670 & 0.190 & 0.710 \\
    POZ & 0.000 & 0.020 & 0.790 & 0.190 & 0.000 & 0.538 \\
    LCJ & 0.990 & 0.010 & 0.000 & 0.000 & 0.000 & 0.136 \\
    GDN & 0.000 & 0.070 & 0.800 & 0.110 & 0.020 & 0.537 \\
    SZZ & 0.960 & 0.040 & 0.000 & 0.000 & 0.000 & 0.147 \\
    BZG & 0.000 & 0.080 & 0.300 & 0.180 & 0.440 & 0.733 \\
    RZE & 0.240 & 0.760 & 0.000 & 0.000 & 0.000 & 0.224 \\
    IEG & 1.000 & 0.000 & 0.000 & 0.000 & 0.000 & 0.010 \\
\hline
\end{tabular}
\caption{Rozkład efektywności dla analizowanych lotnisk}
\label{tab:efficiency-distribution}
\end{table}

\noindent Analiza stochastyczna pozwaliła na zbadanie jakie efektywności osiągają poszczególne lotniska dla różnych, spróbkowanych wartości wag.
Podział na kubełki efektywności pozwala stwierdzić które wartości efektywności są osiągane przez lotniska często, a które rzadko.
Wśród badanych lotnisk wyróżnia się Warszawa Okęcie, dla którego aż 88\% wartości należy do przedziału 0.8-1.0.
W sposób negatywny wyróżniają się lotniska LCJ, SZZ oraz IEG, dla którego 100\% uzyskanych wartości mieści się w przedziale 0.0-0.2.
Dodatkowo wyliczona oszacowana wartość efektywności informuje jaką średnią efektywność uzyskały poszczególne lotniska na wylosowanych próbkach.
Odpowiada ona mniej więcej temu co można zaobserwować na podstawie rozkładu efektywności - najwyższą wartość uzyskuje Warszawa, a najgorsze Łódź, Szczecin oraz Zielona Góra.

\section{Rankingi jednostek}
Rankingi lotnisk uzyskane różnymi metodami (superefektywność, średnia efektywność krzyżowa oraz oczekiwana wartość efektywności).

\noindent Superefektywność: $ WAW \succ GDN \succ BZG \succ KRK \succ WRO \succ POZ \succ KAT \succ RZE \succ LCJ \succ SZZ \succ IEG$ \\
Średnia efektywność krzyżowa: $ BZG \succ GDN \succ WAW \succ WRO \succ KRK \succ POZ \succ KAT \succ RZE \succ LCJ \succ SZZ \succ IEG $ \\
Oczekiwana wartość efektywności: $ WAW \succ BZG \succ WRO \succ KRK \succ POZ \succ GDN \succ KAT \succ RZE \succ SZZ \succ LCJ \succ IEG $ \\

\noindent Uzyskane różnymi sposobami rankingi różnią się miejscami pomiędzy sobą.
Mimo to można zauważyć, że dla wszystkich trzech rankingów w pierwszej trójce lotnisk znajdują się zawsze Warszawa oraz Bydgoszcz.
Gdańsk uzyskuje wysokie pozycje (dwa razy drugie miejsce) dla superefektywności oraz średniej efektywności krzyżowej, ale dopiero szóste miejsce dla oszacowanej efektywności.
W środku rankingów znajdują się zawsze Kraków, Poznań, Wrocław, Katowice oraz Rzeszów chociaż w zależności od rankingu zajmują one różną kolejność.
Wrocław wyróżnia się nieco spośród wymienionych lotnisk, ponieważ dwa razy zajmuje czwartą pozycję (dla efektywności krzyżowej oraz oszacowanej efektywności).
Na końcu rankingu znajdują się zawsze Łódź, Szczecin oraz Zielona Góra. Łódź i Szczecin wymieniają się miejscami w rankingu oszacowanej efektywności i to Szczecin uzyskuje większą wartość.
Zielona Góra dla każdego rankingu wypada najgorzej i zajmuje ostatnie miejsce.

\section{Podsumowanie}

W ramach sprawozdania zbadane zostały efektywności lotnisk w Polsce. Wykorzystano do tego metody Data Envelopment Analysis - wyliczanie efektywności modelem CCR zorientowanym na nakłady, wyliczanie superefektywności, obliczanie efektywności krzyżowej oraz wykorzystanie analizy stochastycznej.
Dla lotnisk, które okazały się nieefektywne wypracowano rekomendacje w których aspektach nakładów muszą się poprawić oraz o ile, aby stać się efektywnymi. Na koniec zaprezentowane zostały rankingi jednostek uzyskane różnymi z wykorzystanych sposobów.

\end{document}